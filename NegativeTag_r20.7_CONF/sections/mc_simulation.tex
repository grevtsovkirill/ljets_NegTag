Samples of inclusive dijet events from strong interaction processes are generated with Pythia 8.186~\cite{pythia} MC generator with the NNPDF 2.3 LO parton distribution functions (PDFs)~\cite{nnpdflo}. This generator utilizes leading-order perturbative quantum chromodynamics (pQCD) matrix elements for $2\rightarrow 2$ processes, along with a leading-logarithmic parton shower~\cite{pythia_ps}, an underlying event (UE) simulation with multiple parton interactions, and the Lund string model for hadronisation~\cite{pythia_hadron}. The parameters for the modelling of the interaction features not represented by the matrix element are provided by the A14 tune~\cite{a14_tune}. Alternative samples of inclusive dijet events from strong interaction processes are generated with HERWIG++ 2.7.1 MC generator~\cite{herwig_pp} with the CTEQ6L1 LO PDFs~\cite{cteqlo} and the UEEE5 tune for the modelling of the interaction feature not represented by the matrix elements (parton shower, hadronization, underlying event).

Generated events are propagated through a full simulation of the ATLAS detector~\cite{SOFT-2010-01} based on Geant4~\cite{geant4} that simulates the particle interactions with the detector material. All generated events are part of the MC15c campaign, which was processed with the release 20.7 of the ATLAS software. Hadronic showers are simulated with the FTFP BERT model~\cite{string_geant4}. Different pileup conditions as a function of the instantaneous luminosity are taken into account by overlaying simulated minimum bias events generated with Pythia8 onto the hard-scattering process and reweighting them according to the distribution of the mean number of interactions observed in data. 

Residual differences between data and simulation regarding jet cleaning requirement efficiencies are at the few percent-level and corrected by means of event-wide scale factors applied to the simulated events. No corrections related to $b$-tagging performance are applied.
