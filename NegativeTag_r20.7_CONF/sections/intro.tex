% Footnote with ATLAS coordinate system
\newcommand{\AtlasCoordFootnote}{%
ATLAS uses a right-handed coordinate system with its origin at the nominal interaction point (IP)
in the centre of the detector and the $z$-axis along the beam pipe.
The $x$-axis points from the IP to the centre of the LHC ring,
and the $y$-axis points upwards.
Cylindrical coordinates $(r,\phi)$ are used in the transverse plane,
$\phi$ being the azimuthal angle around the $z$-axis.
The pseudorapidity is defined in terms of the polar angle $\theta$ as $\eta = -\ln \tan(\theta/2)$.
Angular distance is measured in units of $\Delta R \equiv \sqrt{(\Delta\eta)^{2} + (\Delta\phi)^{2}}$. The transverse energy is defined as $p_{\mathrm{T}} = E / \cosh(\eta)$.}

%---------------------------------------

The identification of jets originating from the hadronization of a $b$-quark ($b$-tagging) is an important element of a number of prominent analyses performed with the ATLAS detector~\cite{PERF-2007-01} at the Large Hadron Collider (LHC): measurements of standard model processes aiming to constrain the heavy-flavour (HF) parton density functions~\cite{photon_hf}, studies of the top quark~\cite{TOPQ-2013-04} and of the Higgs-boson~\cite{Hbb,ttHtobb}, and exploration of New Physics scenarios~\cite{ttbar_MET, EXOT-2016-12}.

The $b$-tagging of a jet relies on the property of the $b$-hadrons to have long lifetime $\tau$ (${\tau\sim1.5}$~ps, corresponding to a proper decay length of about $c\tau\sim$ 450~$\mu$m) and large mass, resulting in the production of tracks with non-zero impact parameters, secondary decay vertices and a large multiplicity of decay products inside the jet cone. These observables are reconstructed with the help of the charged-particle tracking capability of the ATLAS inner detector~\cite{PERF-2012-04}. The information is then combined using a multi-variate algorithm able to enhance the discrimination of a jet containing $b$-hadrons ($b$-jet) with respect to one containing no $b$-hadrons but $c$-hadrons ($c$-jet) or one containing none of them (LF-jet, LF standing for light-flavour). Specific selections on the output weight distribution of a given $b$-tagging algorithm are called working points (WP) and defined as a function of the average efficiency of tagging a $b$-jet as measured in a $t\bar{t}$ simulated sample~\cite{perf_run2,perf_run2_2}. The algorithm most commonly used in the Run 2 ATLAS analyses is called MV2. It assigns to each jet a $b$-tagging discriminant variable ranging from -1 to 1. LF-jets ($b$-jets) MV2 distribution peaks towards -1 (1) whereas the c-jets lie between the two. The 2016 version of MV2, which is the one considered in this document, is trained with a background sample including 7\% of $c$-jets and 93\% of LF-jets and denoted MV2c10 in the following.

The performance of a $b$-tagging algorithm is characterised by the probability of tagging a $b$-jet ($\varepsilon_b$) and the probabilities of mistakenly tagging as a $b$-jet a $c$-jet ($\varepsilon_c$) or a LF-jet ($\varepsilon_l$), referred to as ``mistag rates'' in the following. Ideally, Monte Carlo (MC) simulations including the various quark flavours could be used to evaluate the $b$-tagging performance. However, additional calibration is often needed to account for differences between data and simulation, originating for instance from an imperfect description of the geometry of the detector. In practice, each working point of the algorithm is calibrated as a function of the jet transverse momentum (\ptjet) and absolute pseudorapidity (\aetajet).\footnote{\AtlasCoordFootnote}

This document presents the measurements of the LF-jet mistag rate on ATLAS proton-proton collision data recorded at a center-of-mass energy of $\sqrt{s}=13$~TeV for the MV2c10 WP listed in Table~\ref{tab:wp}~\cite{this_work} using two methods giving consistent results, the negative tag and the adjusted MC. The negative tag method consists in measuring the LF-jet mistag rate from a high statistics data sample enriched in LF-jets with the application of a dedicated algorithm reversing some of the criteria used in the nominal identification algorithm. The adjusted-MC method is based on a bottom-up approach where the underlying tracking variables are adjusted to match the data and the effect is then propagated to the b-tagging observables. Performance plots regarding $b$-jet $b$-tagging rates are available in separate references~\cite{}



\begin{table}[htb]
  \setlength\extrarowheight{3pt}
  \caption{$b$-tagging MV2c10 working points considered in this document. Each WP is defined by a cut value $X$ on the MV2c10 output weight distribution (MV2c10 discrimnant $> X$, MV2c10 discriminant values ranging in $[-1,1]$). The resulting $b$-tagging efficiency ($\effMC_b$), $c$- and LF-jet rejection rates (1/$\varepsilon_c$, 1/$\varepsilon_l$) as measured in a $t\bar{t}$ simulated sample are also shown.}
  \begin{center}
      \begin{tabular}{|c|c | c | cc|}
      \hline
      ~WP~ & ~Cut value $X$~ & ~~$\effMC_b$~~ & ~$c$-jet rejection~ & ~LF-jet rejection~ \\[3.5pt]
      \hline
      85\% &     0.18  &     85\%   & 3   & 34 \\
      77\% &     0.65  &     77\%   & 6   & 134 \\
      70\% &     0.82  &     70\%   & 12  & 381 \\
      60\% &     0.93  &     60\%   & 35  & 1539 \\
      \hline
  \end{tabular}
  \end{center}
\label{tab:wp}
\end{table}


