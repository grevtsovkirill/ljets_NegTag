The proton-proton collision data sample recorded by the ATLAS detector during the year 2015 and 2016 is used. The LHC beams were operated with proton bunches organised in ``bunch train'', with a bunch spacing of 25 ns. Only events taken during stable beam conditions and satisfying detector and data-quality requirements are considered. 

The data used in the measurements were recorded using a suite of single jet triggers~\cite{TRIG-2016-01}, requiring in the event at least one hadronic jet with sufficient transverse energy \ptjet\ and absolute pseudorapidity $\aetajet<3.2$. Hadronic jets are reconstructed from clustered energy deposits~\cite{PERF-2014-07} in the ATLAS calorimeter with the anti-$k_t$ algorithm~\cite{anti_kt} and a parameter $R=0.4$. Given the very high rates of such events at the LHC, only a fraction of events satisfying this requirement were recorded at low and medium \ptjet\ due to computational power and data storage limitations. In order to optimise the statistical power of the measurements, each \ptjet\ bin requires the trigger with lowest prescale (i.e. with highest integrated luminosity) that is more than 99.9\% efficient in that range. The integrated luminosity of the data sample therefore depends on \ptjet\ and ranges from $0.02$~pb$^{-1}$ ($20$~GeV $<\ptjet<60$~GeV) to $36.1$~fb$^{-1}$ ($\ptjet>500$~GeV).

Events are required to have at least one reconstructed vertex with at least two associated well-reconstructed tracks. Futhermore, at least two jets reconstructed within the ATLAS inner detector pseudorapidity acceptance ($\aetajet<2.5$) passing cleaning criteria~\cite{ATLAS-CONF-2015-029}, identified as coming from the primary hard interaction~\cite{ATLAS-CONF-2014-018} and satisfying $\ptjet>20$~GeV after final calibration~\cite{PERF-2016-04} must be present. If more than two jets satisfy these criteria, the two jets with the highest transverse momenta are selected and the others are disregarded. A good angular separation between the two jets in the transverse plane ($\Delta\phi_{jj}>2$~rad.) is also required in order to reject events with high transverse momentum jets originating from the hadronization of a gluon which split into two quarks ($g\rightarrow q\bar{q}$), more likely to contain $c$- and $b$-jets, or beam-induced background due to proton losses upstream of the interaction point~\cite{DAPR-2012-01}. 

