The identification of $b$-jets in the ATLAS experiment is based on a set of $b$-tagging algorithms relying on observables based on the reconstruction of signed track impact-parameters, signed secondary
vertices, and decay chain multi-vertices. The output of the $b$-tagging algorithms is then combined into MV2c10, a multivariate discriminant trained on a background sample including 7\% of $c$-jet and 93\% of LF-jets for the optimal separation between $b$-jets and $c$-jets/LF-jets, with values ranging between -1 and 1. LF-jets (b-jets) MV2 distribution peaks towards -1 (1) whereas the c-jets lie between the two. A detailed description about all the tagging algorithms used in MV2 can be found in~\cite{perf_run2,perf_run2_2}. They are described briefly below.

\begin{itemize}

\item \textbf{Impact Parameter based Algorithms (IP2D and IP3D):} due to long lifetime of $b$-hadrons, tracks generated from $b$-hadron decay products tend to have large impact parameters enabling their contribution to be separated from the contribution of tracks from the primary vertex. The IP2D and IP3D algorithms make use of the signed impact parameter significance of the tracks associated with the jet. The sign is defined positive (negative) if the point of closest approach of the track to the primary vertex is in front (behind) the primary vertex with respect to the jet direction. IP3D uses both the transverse and longitudinal impact parameters taking into account their correlations, while IP2D uses the transverse impact parameters only. The probability density functions (pdfs) for the signed impact parameter significance of these tracks are used to define ratios of the b, c and light-flavour jet hypotheses, and these are then combined in three log likelihood ratio discriminant (LLR): $b$/LF, $b$/$c$ and $c$/LF.
 
\item \textbf{Secondary Vertex Finding Algorithm (SV1):} SV1 aims to explicitly reconstruct an inclusive displaced secondary vertex within the jet. The first step consists of reconstructing two-track vertices using the candidate tracks. Tracks are rejected if they form a secondary vertex which can be identified as likely originating from the decay of a long-lived particle (e.g. $K_s$ or $\Lambda$), photon conversions or hadronic interactions with the detector material. A new vertex is then fitted with all tracks surviving this selection, outlier tracks being iteratively removed from this set of tracks. The vertex information is condensed in the eight following observables: the invariant mass of tracks at the secondary vertex assuming pion masses, the fraction of the charged jet energy in the secondary vertex, the number of tracks used in the secondary vertex, the number of two track vertex candidates, the transverse distance between the primary and secondary vertices, the distance between the primary and secondary vertices, the distance between the primary and secondary vertices divided by its uncertainty and the $\Delta R$ between the jet axis and the direction of the secondary vertex relative to the primary vertex. 

\item \textbf{Decay Chain Multi-Vertex Algorithm (JetFitter):} JetFitter exploits the topological structure of weak $b$- and $c$-hadron decays inside the jet and tries to reconstruct the full $b$-hadron decay chain. A Kalman filter is used to find a common line on which the primary vertex and the HF vertices lie, approximating the $b$-hadron flight path, as well as their positions. Hence, HF vertices can be resolved even when only a single track is attached to them whenever the resolution allows. The vertex information is condensed in the eight following observables: the number of 2-track vertex candidates prior to the decay chain fit, the invariant mass of tracks from displaced vertices assuming pion masses, the significance of the average distance between the primary and displaced vertices, the fraction of the charged jet energy in the secondary vertices, the number of displaced vertices with one track, the number of displaced vertices with more than one track, the number of tracks from displaced vertices with at least two tracks and the $\Delta R$ between the jet axis and the vectorial sum of the momenta of all tracks attached to displaced vertices.

\end{itemize}
