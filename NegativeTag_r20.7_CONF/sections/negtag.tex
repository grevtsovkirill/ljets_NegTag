The negative tag method~\cite{this_work} relies to a large extent on the assumption that LF-jets are mistagged as $b$-jets mainly because of the finite resolution of the inner detector track impact parameters. Due to the long lifetime of $b$- and $c$-hadrons, the signed impact parameter distribution of the tracks associated with $b$- and $c$-hadron decays (i.e. within $b$- and $c$-jets) is expected to show a high tail at large positive values. This is the main discriminating feature used in impact parameter based $b$-tagging algorithms. Assuming that the LF-jet mistag is driven by resolution effects, the signed impact parameter distribution of the tracks associated with LF-jets, i.e. jets containing no $b$- nor $c$-hadrons, is expected to be symmetric. These properties are observed in simulated events, as shown in Figure~\ref{fig1}. The inclusive tag rate obtained by re-running the MV2c10 algorithm after reversing the impact parameter significance sign of tracks is therefore expected to be a good approximation of the LF-mistag rate for impact parameter based algorithms. Moreover, the $b$-, $c$- and LF-jets impact parameter distributions being much more similar on the negative side than on the positive side (see Figure~\ref{fig1}), one expects the inclusive negative tag rate to be comparable for the three flavors. Similar features are expected for the signed decay length significance of secondary vertices reconstructed with SV1 and JetFitter, which are seeded from tracks. 

A dedicated tagging algorithm aiming to tag jets including negatively displaced tracks and negative lifetime secondary vertices, denoted MV2c10Flip in the following, is defined. It will be used as proxy to the LF-mistag rate. MV2c10Flip uses the same training and includes the same input variables as MV2c10, however the list of inputs variables is taken from modified versions of IP2D, IP3D, SV1 and JetFitter called respectively IP2DNeg, IP3DNeg, SV1Flip and JetFitterFlip, defined as follow: 

\begin{itemize}

\item \textbf{Negative Impact Parameter based Algorithms (IP2DNeg, IP3DNeg):} only tracks with negative transverse impact parameter (d$_{0}$) are selected as inputs. The d$_{0}$ sign of the selected tracks is flipped before running the IP2D/IP3D algorithms. In the standard versions of IP2D/IP3D, both positive and negative d$_{0}$ tracks are used as inputs and no flipping is performed. 

\item \textbf{Flipped Secondary Vertex Finding Algorithm (SV1Flip):} the d$_{0}$ sign of the track used as inputs is flipped before running the SV1 algorithm. Once defined by the algorithm, only the two-track vertices with negative lifetime are considered to compute the output distributions. In the standard version of SV1, no flipping is performed and only two-track vertices with positive lifetime are considered to compute the output distributions.

\item \textbf{Flipped Decay Chain Multi-Vertex Algorithm (JetFitterFlip):} the d$_{0}$ sign of the track used as inputs for two-track vertex reconstruction is flipped before the JetFitter algorithm. The sign of the two-track vertex candidates used to seed tracks for JetFitter is also flipped before running the rest of the algorithm. Once defined by the algorithm, only the final vertices with negative lifetime are considered to compute the output distributions. In the standard version of JetFitter, no flipping is performend and only final vertices with positive lifetime are considered to compute the output distributions. 

\end{itemize}

A jet is then considered negatively tagged if the MV2c10Flip tag discriminant variable, ranging also between -1 and 1, satisfies the nominal WP cut value defined earlier in Table~\ref{tab:wp}. 
