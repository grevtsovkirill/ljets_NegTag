%-------------------------------------------------------------------------------
% This file provides a skeleton ATLAS paper.
%-------------------------------------------------------------------------------
% \pdfoutput=1
% The \pdfoutput command is needed by arXiv/JHEP/JINST to ensure use of pdflatex.
% It should be included in the first 5 lines of the file.
% \pdfinclusioncopyfonts=1
% This command may be needed in order to get \ell in PDF plots to appear. Found in
% https://tex.stackexchange.com/questions/322010/pdflatex-glyph-undefined-symbols-disappear-from-included-pdf
%-------------------------------------------------------------------------------
% Specify where ATLAS LaTeX style files can be found.
\newcommand*{\ATLASLATEXPATH}{latex/}
% Use this variant if the files are in a central location, e.g. $HOME/texmf.
% \newcommand*{\ATLASLATEXPATH}{}
%-------------------------------------------------------------------------------
\documentclass[CONF, UKenglish, texlive=2016, coverpage]{\ATLASLATEXPATH atlasdoc}
% The language of the document must be set: usually UKenglish or USenglish.
% british and american also work!
% Commonly used options:
%  texlive=YYYY          Specify TeX Live version (2016 is default).
%  coverpage             Create ATLAS draft cover page for collaboration circulation.
%                        See atlas-draft-cover.tex for a list of variables that should be defined.
%  cernpreprint          Create front page for a CERN preprint.
%                        See atlas-preprint-cover.tex for a list of variables that should be defined.
%  NOTE                  The document is an ATLAS note (draft).
%  PAPER                 The document is an ATLAS paper (draft).
%  CONF                  The document is a CONF note (draft).
%  PUB                   The document is a PUB note (draft).
%  BOOK                  The document is of book form, like an LOI or TDR (draft)
%  txfonts=true|false    Use txfonts rather than the default newtx
%  paper=a4|letter       Set paper size to A4 (default) or letter.

%-------------------------------------------------------------------------------
% Extra packages:
\usepackage{\ATLASLATEXPATH atlaspackage}
% Commonly used options:
%  biblatex=true|false   Use biblatex (default) or bibtex for the bibliography.
%  backend=bibtex        Use the bibtex backend rather than biber.
%  subfigure|subfig|subcaption  to use one of these packages for figures in figures.
%  minimal               Minimal set of packages.
%  default               Standard set of packages.
%  full                  Full set of packages.
%-------------------------------------------------------------------------------
% Style file with biblatex options for ATLAS documents.
\usepackage{\ATLASLATEXPATH atlasbiblatex}

% Useful macros
\usepackage{\ATLASLATEXPATH atlasphysics}
% See doc/atlas_physics.pdf for a list of the defined symbols.
% Default options are:
%   true:  journal, misc, particle, unit, xref
%   false: BSM, heppparticle, hepprocess, hion, jetetmiss, math, process, other, texmf
% See the package for details on the options.

% Files with references for use with biblatex.
% Note that biber gives an error if it finds empty bib files.
\addbibresource{LFCalib_conf.bib}
\addbibresource{bib/ATLAS.bib}
\addbibresource{bib/CMS.bib}
\addbibresource{bib/ConfNotes.bib}
\addbibresource{bib/PubNotes.bib}

% Paths for figures - do not forget the / at the end of the directory name.
\graphicspath{{logos/}{figures/}}

% Add you own definitions here (file LFCalib_conf-defs.sty).
\usepackage{LFCalib_conf-defs}

%-------------------------------------------------------------------------------
% Generic document information
%-------------------------------------------------------------------------------

% Title, abstract and document 
%-------------------------------------------------------------------------------
% This file contains the title, author and abstract.
% It also contains all relevant document numbers used by the different cover pages.
%-------------------------------------------------------------------------------

% Title
\AtlasTitle{Calibration of light-flavour jet $b$-tagging rates on ATLAS proton-proton collision data at $\sqrt{s}=13$~TeV}

% Draft version:
% Should be 1.0 for the first circulation, and 2.0 for the second circulation.
% If given, adds draft version on front page, a 'DRAFT' box on top of each other page, 
% and line numbers.
% Comment or remove in final version.
\AtlasVersion{0.1}

% Abstract - % directly after { is important for correct indentation
\AtlasAbstract{%
A variety of algorithms have been developed to identify jets originating from $b$-quark hadronization within the ATLAS experiment at the Large Hadron Collider. We describe two measurements of the misidentification rate of jets containing no $b$- nor $c$-hadrons for the algorithm most commonly used in the LHC Run~2 ATLAS analyses. The measurements are performed in various ranges of jet transverse momenta and pseudorapidities based on proton-proton collision data collected at a centre-of-mass energy of ${\sqrt{s} = 13}$~TeV during the year 2015 and 2016. The first measurement is based on a data sample enriched in jets originating from light-flavour quark and gluon hadronization with the application of a dedicated algorithm reversing some of the criteria used in the nominal identification algorithm. The second measurement is based on a bottom-up approach where the underlying tracking variables in the simulation are adjusted to match the data. The effect is then propagated to the high-level observables relevant for $b$-identification. The results of both methods are found in good agreement and compared to the misidentification rate predicted by the nominal ATLAS simulation in order to calibrate it.
}

% Author - this does not work with revtex (add it after \begin{document})
\author{The ATLAS Collaboration}

% ATLAS reference code, to help ATLAS members to locate the paper
\AtlasRefCode{CONF-FTAG-2017-02}

% CERN preprint number
% \PreprintIdNumber{CERN-PH-2016-XX}

% ATLAS date - arXiv submission; usually filled in by the Physics Office
% \AtlasDate{\today}

% ATLAS heading - heading at top of title page. Set for TDR etc.
% \AtlasHeading{ATLAS ABC TDR}

% arXiv identifier
% \arXivId{14XX.YYYY}

% HepData record
% \HepDataRecord{ZZZZZZZZ}

% Submission journal and final reference
% \AtlasJournal{Phys.\ Lett.\ B.}
% \AtlasJournalRef{\PLB 789 (2014) 123}
% \AtlasDOI{}

%-------------------------------------------------------------------------------
% The following information is needed for the cover page. The commands are only defined
% if you use the coverpage option in atlasdoc or use the atlascover package
%-------------------------------------------------------------------------------

% List of supporting notes  (leave as null \AtlasCoverSupportingNote{} if you want to skip this option)
\AtlasCoverSupportingNote{ATL-COM-PHYS-2017-192}{https://cds.cern.ch/record/2253746}
% \AtlasCoverSupportingNote{Short title note 2}{https://cds.cern.ch/record/YYYYYYY}
%
% OR (the 2nd option is deprecated, especially for CONF and PUB notes)
%
% Supporting material TWiki page  (leave as null \AtlasCoverTwikiURL{} if you want to skip this option)
% \AtlasCoverTwikiURL{https://twiki.cern.ch/twiki/bin/view/Atlas/WebHome}

% Comment deadline
\AtlasCoverCommentsDeadline{DD Month 2017}

% Analysis team members - contact editors should no longer be specified
% as there is a generic email list name for the editors
\AtlasCoverAnalysisTeam{Valentina Cairo, Stefano Cali, Julian Glatzer, Emily Graham, Kristian Gregersen, David Jamin, Krisztian Peters, Matthias Saimpert, Balthasar Schachtner, Federico Sforza, Jonathan Shlomi}

% Editorial Board Members - indicate the Chair by a (chair) after his/her name
% Give either all members at once (then they appear on one line), or separately
 \AtlasCoverEdBoardMember{Paolo Francavilla (chair), Claudia Gemme, Vadim Kostyukhin}

% Editors egroup
\AtlasCoverEgroupEditors{atlas-conf-ftag-2017-02-editors@cern.ch}

% EdBoard egroup
\AtlasCoverEgroupEdBoard{atlas-conf-ftag-2017-02-editorial-board@cern.ch}

% Author and title for the PDF file
\hypersetup{pdftitle={ATLAS document},pdfauthor={The ATLAS Collaboration}}

%-------------------------------------------------------------------------------
% Content
%-------------------------------------------------------------------------------
\begin{document}

\maketitle

\tableofcontents


%-------------------------------------------------------------------------------
\section{Introduction}
\label{sec:intro}
%-------------------------------------------------------------------------------

% Footnote with ATLAS coordinate system
\newcommand{\AtlasCoordFootnote}{%
ATLAS uses a right-handed coordinate system with its origin at the nominal interaction point (IP)
in the centre of the detector and the $z$-axis along the beam pipe.
The $x$-axis points from the IP to the centre of the LHC ring,
and the $y$-axis points upwards.
Cylindrical coordinates $(r,\phi)$ are used in the transverse plane,
$\phi$ being the azimuthal angle around the $z$-axis.
The pseudorapidity is defined in terms of the polar angle $\theta$ as $\eta = -\ln \tan(\theta/2)$.
Angular distance is measured in units of $\Delta R \equiv \sqrt{(\Delta\eta)^{2} + (\Delta\phi)^{2}}$. The transverse energy is defined as $p_{\mathrm{T}} = E / \cosh(\eta)$.}

%---------------------------------------

The identification of jets originating from the hadronization of a $b$-quark ($b$-tagging) is an important element of a number of prominent analyses performed with the ATLAS detector~\cite{PERF-2007-01} at the Large Hadron Collider (LHC): measurements of standard model processes aiming to constrain the heavy-flavour (HF) parton density functions~\cite{photon_hf}, studies of the top quark~\cite{TOPQ-2013-04} and of the Higgs-boson~\cite{Hbb,ttHtobb}, and exploration of New Physics scenarios~\cite{ttbar_MET, EXOT-2016-12}.

The $b$-tagging of a jet relies on the property of the $b$-hadrons to have long lifetime $\tau$ (${\tau\sim1.5}$~ps, corresponding to a proper decay length of about $c\tau\sim$ 450~$\mu$m) and large mass, resulting in the production of tracks with non-zero impact parameters, secondary decay vertices and a large multiplicity of decay products inside the jet cone. These observables are reconstructed with the help of the charged-particle tracking capability of the ATLAS inner detector~\cite{PERF-2012-04}. The information is then combined using a multi-variate algorithm able to enhance the discrimination of a jet containing $b$-hadrons ($b$-jet) with respect to a jet containing no $b$-hadrons but $c$-hadrons ($c$-jet) or a jet containing no $b$-hadrons nor $c$-hadrons (LF-jet, LF standing for light-flavour). Specific selections on the output weight distribution of a given $b$-tagging algorithm are called working points (WP) and defined as a function of the average efficiency of tagging a $b$-jet as measured in a $t\bar{t}$ simulated sample~\cite{perf_run2,perf_run2_2}. The algorithm most commonly used in the Run 2 ATLAS analyses is called MV2. It assigns to each jet a $b$-tagging discriminant variable ranging from -1 to 1. LF-jets ($b$-jets) MV2 distribution peaks towards -1 (1) whereas the c-jets lie between the two. The 2016 version of MV2, which is the one considered in this document, is trained with a background sample including 7\% of $c$-jets and 93\% of LF-jets. It is denoted MV2c10 in the following.

The performance of a $b$-tagging algorithm is characterised by the probability of tagging a $b$-jet ($\varepsilon_b$) and the probabilities of mistakenly tagging as a $b$-jet a $c$-jet ($\varepsilon_c$) or a LF-jet ($\varepsilon_l$), referred to as ``mistag rates'' in the following. Ideally, Monte Carlo (MC) simulations including the various quark flavours could be used to evaluate the $b$-tagging performance. However, additional calibration is often needed to account for differences between data and simulation, originating for instance from an imperfect description of the geometry of the detector. In practice, each working point of the algorithm is calibrated as a function of the jet transverse momentum (\ptjet) and absolute pseudorapidity (\aetajet).\footnote{\AtlasCoordFootnote}

This document presents the measurements of the LF-jet mistag rate on ATLAS proton-proton collision data recorded at a center-of-mass energy of $\sqrt{s}=13$~TeV for the MV2c10 WP listed in Table~\ref{tab:wp}~\cite{this_work} using two methods giving consistent results, the negative tag and the adjusted MC. The negative tag method consists in measuring the LF-jet mistag rate from a high statistics data sample enriched in LF-jets with the application of a dedicated algorithm reversing some of the criteria used in the nominal identification algorithm. The adjusted-MC method is based on a bottom-up approach where the underlying tracking variables are adjusted to match the data and the effect is then propagated to the b-tagging observables. $b$-jet performance are described in separate references~\cite{calib1,calib2,calib3}.



\begin{table}[htb]
  \setlength\extrarowheight{3pt}
  \caption{$b$-tagging MV2c10 working points considered in this document. Each WP is defined by a cut value $X$ on the MV2c10 output weight distribution (MV2c10 discrimnant $> X$, MV2c10 discriminant values ranging in $[-1,1]$). The resulting $b$-tagging efficiency ($\effMC_b$) and $c$- and LF-jet rejection rates (1/$\varepsilon_c$, 1/$\varepsilon_l$) as measured in a $t\bar{t}$ simulated sample are also shown.}
  \begin{center}
      \begin{tabular}{|c|c | c | cc|}
      \hline
      ~WP~ & ~Cut value $X$~ & ~~$\effMC_b$~~ & ~$c$-jet rejection~ & ~LF-jet rejection~ \\[3.5pt]
      \hline
      85\% &     0.18  &     85\%   & 3   & 34 \\
      77\% &     0.65  &     77\%   & 6   & 134 \\
      70\% &     0.82  &     70\%   & 12  & 381 \\
      60\% &     0.93  &     60\%   & 35  & 1539 \\
      \hline
  \end{tabular}
  \end{center}
\label{tab:wp}
\end{table}




%-------------------------------------------------------------------------------
\section{ATLAS detector}
\label{sec:detector}
%-------------------------------------------------------------------------------

The ATLAS experiment~\cite{PERF-2007-01} at the LHC is a multi-purpose particle detector
with a forward-backward symmetric cylindrical geometry and a near $4\pi$ coverage in 
solid angle.
It consists of an inner tracking detector surrounded by a thin superconducting solenoid
providing a \SI{2}{\tesla} axial magnetic field, electromagnetic and hadron calorimeters, and a muon spectrometer.

The inner tracking detector (ID) covers the pseudorapidity range $|\eta| < 2.5$.
It consists of silicon pixel, silicon micro-strip, and transition radiation tracking detectors.
Among them, the pixel detector is crucial for b-jet identification. A new inner pixel layer, the Insertable B-Layer~\cite{IBL, IBL_perf} (IBL), was added before the start of 
Run 2, at a mean sensor radius of 3.2 cm from the beam-line. 

Outside the ID, the lead/liquid-argon (LAr) sampling calorimeters provide electromagnetic (EM) energy measurements
with high granularity.
A hadron (iron/scintillator-tile) calorimeter covers the central pseudorapidity range ($|\eta| < 1.7$).
The end-cap and forward regions are instrumented with LAr calorimeters
for both EM and hadronic energy measurements up to $|\eta| = 4.9$.
The muon spectrometer surrounds the calorimeters and is based on
three large air-core toroid superconducting magnets with eight coils each.
Its bending power is in the range from \num{2.0} to \SI{7.5}{\tesla\metre}.
It includes a system of precision tracking chambers and fast detectors for triggering.

A two-level trigger system, using custom
hardware followed by a software-based level, is used to reduce the event storage rate to a maximum of
around 1 kHz.


%-------------------------------------------------------------------------------
\section{Data sample and event selection}
\label{sec:data_sample}
%-------------------------------------------------------------------------------

The proton-proton collision data sample recorded by the ATLAS detector during the year 2015 and 2016 is used. The LHC beams were operated with proton bunches organised in ``bunch train'', with a bunch spacing of 25 ns. Only events taken during stable beam conditions and satisfying detector and data-quality requirements are considered. 

The data used in the measurements were recorded using a suite of single jet triggers~\cite{TRIG-2016-01}, requiring in the event at least one hadronic jet with sufficient transverse energy \ptjet\ and absolute pseudorapidity $\aetajet<3.2$. Hadronic jets are reconstructed from clustered energy deposits~\cite{PERF-2014-07} in the ATLAS calorimeter with the anti-$k_t$ algorithm~\cite{anti_kt} and a parameter $R=0.4$. Given the very high rates of such events at the LHC, only a fraction of events satisfying this requirement were recorded at low and medium \ptjet\ due to computational power and data storage limitations. In order to optimise the statistical power of the measurements, each \ptjet\ bin requires the trigger with lowest prescale (i.e. with highest integrated luminosity) that is more than 99.9\% efficient in that range. The integrated luminosity of the data sample therefore depends on \ptjet\ and ranges from $0.02$~pb$^{-1}$ ($20$~GeV $<\ptjet<60$~GeV) to $36.1$~fb$^{-1}$ ($\ptjet>500$~GeV).

Events are required to have at least one reconstructed vertex with at least two associated well-reconstructed tracks. Futhermore, at least two jets reconstructed within the ATLAS inner detector pseudorapidity acceptance ($\aetajet<2.5$) passing cleaning criteria~\cite{ATLAS-CONF-2015-029}, identified as coming from the primary hard interaction~\cite{ATLAS-CONF-2014-018} and satisfying $\ptjet>20$~GeV after final calibration~\cite{PERF-2016-04} must be present. If more than two jets satisfy these criteria, the two jets with the highest transverse momenta are selected and the others are disregarded. A good angular separation between the two jets in the transverse plane ($\Delta\phi_{jj}>2$~rad.) is also required in order to reject events with high transverse momentum jets originating from the hadronization of a gluon which split into two quarks ($g\rightarrow q\bar{q}$), more likely to contain $c$- and $b$-jets, or beam-induced background due to proton losses upstream of the interaction point~\cite{DAPR-2012-01}. 



%-------------------------------------------------------------------------------
\section{Monte Carlo simulation}
\label{sec:mc_simulation}
%-------------------------------------------------------------------------------

Samples of inclusive dijet events from strong interaction processes are generated with Pythia 8.186~\cite{pythia} MC generator with the NNPDF 2.3 LO parton distribution functions (PDFs)~\cite{nnpdflo}. This generator utilizes leading-order perturbative quantum chromodynamics (pQCD) matrix elements for $2\rightarrow 2$ processes, along with a leading-logarithmic parton shower~\cite{pythia_ps}, an underlying event (UE) simulation with multiple parton interactions, and the Lund string model for hadronisation~\cite{pythia_hadron}. The parameters for the modelling of the interaction features not represented by the matrix element are provided by the A14 tune~\cite{a14_tune}. Alternative samples of inclusive dijet events from strong interaction processes are generated with HERWIG++ 2.7.1 MC generator~\cite{herwig_pp} with the CTEQ6L1 LO PDFs~\cite{cteqlo} and the UEEE5 tune for the modelling of the interaction feature not represented by the matrix elements (parton shower, hadronization, underlying event).

Generated events are propagated through a full simulation of the ATLAS detector~\cite{SOFT-2010-01} based on Geant4~\cite{geant4} that simulates the particle interactions with the detector material. All generated events are part of the MC15c campaign, which was processed with the release 20.7 of the ATLAS software. Hadronic showers are simulated with the FTFP BERT model~\cite{string_geant4}. Different pileup conditions as a function of the instantaneous luminosity are taken into account by overlaying simulated minimum bias events generated with Pythia8 onto the hard-scattering process and reweighting them according to the distribution of the mean number of interactions observed in data. 

Residual differences between data and simulation regarding jet cleaning requirement efficiencies are at the few percent-level and corrected by means of event-wide scale factors applied to the simulated events. No corrections related to $b$-tagging performance are applied to MC simulation for the measurements.


%-------------------------------------------------------------------------------
\section{$b$-tagging algorithms}
\label{sec:algo}
%-------------------------------------------------------------------------------

The identification of $b$-jets in the ATLAS experiment is based on a set of $b$-tagging algorithms relying on observables based on the reconstruction of signed track impact-parameters, signed secondary
vertices, and decay chain multi-vertices. The output of the $b$-tagging algorithms is then combined into MV2c10, a multivariate discriminant trained on a background sample including 7\% of $c$-jet and 93\% of LF-jets for the optimal separation between $b$-jets and $c$-jets/LF-jets, with values ranging between -1 and 1. LF-jets (b-jets) MV2 distribution peaks towards -1 (1) whereas the c-jets lie between the two. A detailed description about all the tagging algorithms used in MV2 can be found in~\cite{perf_run2,perf_run2_2}. They are described briefly below.

\begin{itemize}

\item \textbf{Impact Parameter based Algorithms (IP2D and IP3D):} due to long lifetime of $b$-hadrons, tracks generated from $b$-hadron decay products tend to have large impact parameters enabling their contribution to be separated from the contribution of tracks from the primary vertex. The IP2D and IP3D algorithms make use of the signed impact parameter significance of the tracks associated with the jet. The sign is defined positive (negative) if the point of closest approach of the track to the primary vertex is in front (behind) the primary vertex with respect to the jet direction. IP3D uses both the transverse and longitudinal impact parameters taking into account their correlations, while IP2D uses the transverse impact parameters only. The probability density functions (pdfs) for the signed impact parameter significance of these tracks are used to define ratios of the b, c and light-flavour jet hypotheses, and these are then combined in three log likelihood ratio discriminant (LLR): $b$/LF, $b$/$c$ and $c$/LF.
 
\item \textbf{Secondary Vertex Finding Algorithm (SV1):} SV1 aims to explicitly reconstruct an inclusive displaced secondary vertex within the jet. The first step consists of reconstructing two-track vertices using the candidate tracks. Tracks are rejected if they form a secondary vertex which can be identified as likely originating from the decay of a long-lived particle (e.g. $K_s$ or $\Lambda$), photon conversions or hadronic interactions with the detector material. A new vertex is then fitted with all tracks surviving this selection, outlier tracks being iteratively removed from this set of tracks. The vertex information is condensed in the eight following observables: the invariant mass of tracks at the secondary vertex assuming pion masses, the fraction of the charged jet energy in the secondary vertex, the number of tracks used in the secondary vertex, the number of two track vertex candidates, the transverse distance between the primary and secondary vertices, the distance between the primary and secondary vertices, the distance between the primary and secondary vertices divided by its uncertainty and the $\Delta R$ between the jet axis and the direction of the secondary vertex relative to the primary vertex. 

\item \textbf{Decay Chain Multi-Vertex Algorithm (JetFitter):} JetFitter exploits the topological structure of weak $b$- and $c$-hadron decays inside the jet and tries to reconstruct the full $b$-hadron decay chain. A Kalman filter is used to find a common line on which the primary vertex and the HF vertices lie, approximating the $b$-hadron flight path, as well as their positions. Hence, HF vertices can be resolved even when only a single track is attached to them whenever the resolution allows. The vertex information is condensed in the eight following observables: the number of 2-track vertex candidates prior to the decay chain fit, the invariant mass of tracks from displaced vertices assuming pion masses, the significance of the average distance between the primary and displaced vertices, the fraction of the charged jet energy in the secondary vertices, the number of displaced vertices with one track, the number of displaced vertices with more than one track, the number of tracks from displaced vertices with at least two tracks and the $\Delta R$ between the jet axis and the vectorial sum of the momenta of all tracks attached to displaced vertices.

\end{itemize}


%-------------------------------------------------------------------------------
\section{Calibration with the negative tag method}
\label{sec:negatag}
%-------------------------------------------------------------------------------

The negative tag method~\cite{this_work} relies to a large extent on the assumption that LF-jets are mistagged as $b$-jets mainly because of the finite resolution of the inner detector track impact parameters. Due to the long lifetime of $b$- and $c$-hadrons, the signed impact parameter distribution of the tracks associated with $b$- and $c$-hadron decays (i.e. within $b$- and $c$-jets) is expected to show a high tail at large positive values. This is the main discriminating feature used in impact parameter based $b$-tagging algorithms. Assuming that the LF-jet mistag is driven by resolution effects, the signed impact parameter distribution of the tracks associated with LF-jets, i.e. jets containing no $b$- nor $c$-hadrons, is expected to be symmetric. These properties are observed in simulated events, as shown in Figure~\ref{fig1}. The inclusive tag rate obtained by re-running the MV2c10 algorithm after reversing the impact parameter significance sign of tracks is therefore expected to be a good approximation of the LF-mistag rate for impact parameter based algorithms. Moreover, the $b$-, $c$- and LF-jets impact parameter distributions being much more similar on the negative side than on the positive side (see Figure~\ref{fig1}), one expects the inclusive negative tag rate to be comparable for the three flavors. Similar features are expected for the signed decay length significance of secondary vertices reconstructed with SV1 and JetFitter, which are seeded from tracks. 

A dedicated tagging algorithm aiming to tag jets including negatively displaced tracks and negative lifetime secondary vertices, denoted MV2c10Flip in the following, is defined. It will be used as proxy to the LF-mistag rate. MV2c10Flip uses the same training and includes the same input variables as MV2c10, however the list of inputs variables is taken from modified versions of IP2D, IP3D, SV1 and JetFitter called respectively IP2DNeg, IP3DNeg, SV1Flip and JetFitterFlip, defined as follow: 

\begin{itemize}

\item \textbf{Negative Impact Parameter based Algorithms (IP2DNeg, IP3DNeg):} only tracks with negative transverse impact parameter (d$_{0}$) are selected as inputs. The d$_{0}$ sign of the selected tracks is flipped before running the IP2D/IP3D algorithms. In the standard versions of IP2D/IP3D, both positive and negative d$_{0}$ tracks are used as inputs and no flipping is performed. 

\item \textbf{Flipped Secondary Vertex Finding Algorithm (SV1Flip):} the d$_{0}$ sign of the track used as inputs is flipped before running the SV1 algorithm. Once defined by the algorithm, only the two-track vertices with negative lifetime are considered to compute the output distributions. In the standard version of SV1, no flipping is performed and only two-track vertices with positive lifetime are considered to compute the output distributions.

\item \textbf{Flipped Decay Chain Multi-Vertex Algorithm (JetFitterFlip):} the d$_{0}$ sign of the track used as inputs for two-track vertex reconstruction is flipped before the JetFitter algorithm. The sign of the two-track vertex candidates used to seed tracks for JetFitter is also flipped before running the rest of the algorithm. Once defined by the algorithm, only the final vertices with negative lifetime are considered to compute the output distributions. In the standard version of JetFitter, no flipping is performend and only final vertices with positive lifetime are considered to compute the output distributions. 

\end{itemize}

A jet is then considered negatively tagged if the MV2c10Flip tag discriminant variable, ranging also between -1 and 1, satisfies the nominal WP cut value defined earlier in Table~\ref{tab:wp}. 


%You can find some text snippets that can be used in papers in \texttt{template/atlas-snippets.tex}.
%Some of the snippets need the \texttt{jetetmiss} option passed to \texttt{atlasphysics}.
%\subsection{\Antikt}

The \antikt algorithm with a radius parameter of $R=0.4$ is used to reconstruct jets with a four-momentum recombination scheme, using \topos as inputs. Jet energy is calibrated to the hadronic scale with the effect of \pileup removed

\subsection{\Topos}

Hadronic jets are reconstructed from calibrated three-dimensional \topos.
Clusters are constructed from calorimeter cells that are grouped together using a topological clustering algorithm.
These objects provide a three-dimensional representation of energy depositions in the calorimeter and implement a nearest-neighbour noise suppression algorithm.
The resulting \topos are classified as either electromagnetic or hadronic based on their shape, depth and energy density.
Energy corrections are then applied to the clusters in order to calibrate them to the appropriate energy scale for their classification.
These corrections are collectively referred to as \textit{local cluster weighting}, or LCW, and jets that are calibrated using this procedure are referred to as LCW jets~\cite{PERF-2012-01}.

\subsection{Grooming}

Trimming removes subjets with $\ptsubji/\ptjet < \fcut$, where \ptsubji is the transverse momentum of the $i^{\text{th}}$ subjet, and $\fcut=0.05$.
Filtering proceeds similarly, but utilises the relative masses of the subjets defined and the original jet. For at least one of the configurations tested, trimming and filtering are both able to approximately eliminate the \pileup dependence of the jet mass.

%-------------------------------------------------------------------------------
\section{Calibration with the adjusted Monte Carlo method}
\label{sec:adjMC}
%-------------------------------------------------------------------------------

\input{sections/adjMC.tex}

%-------------------------------------------------------------------------------
\section{Comparison between the two methods}
\label{sec:comparison}
%-------------------------------------------------------------------------------

\input{sections/comparison.tex}

% All figures and tables should appear before the summary and conclusion.
% The package placeins provides the macro \FloatBarrier to achieve this.
% \FloatBarrier

%-------------------------------------------------------------------------------
\section{Conclusion}
\label{sec:conclusion}
%-------------------------------------------------------------------------------

\input{sections/conclusion.tex}

%-------------------------------------------------------------------------------
\section*{Acknowledgements}
%-------------------------------------------------------------------------------

%% Acknowledgements for papers with collision data
% Version 06-March-2017

% Standard acknowledgements start here
%----------------------------------------------
We thank CERN for the very successful operation of the LHC, as well as the
support staff from our institutions without whom ATLAS could not be
operated efficiently.

We acknowledge the support of ANPCyT, Argentina; YerPhI, Armenia; ARC, Australia; BMWFW and FWF, Austria; ANAS, Azerbaijan; SSTC, Belarus; CNPq and FAPESP, Brazil; NSERC, NRC and CFI, Canada; CERN; CONICYT, Chile; CAS, MOST and NSFC, China; COLCIENCIAS, Colombia; MSMT CR, MPO CR and VSC CR, Czech Republic; DNRF and DNSRC, Denmark; IN2P3-CNRS, CEA-DSM/IRFU, France; SRNSF, Georgia; BMBF, HGF, and MPG, Germany; GSRT, Greece; RGC, Hong Kong SAR, China; ISF, I-CORE and Benoziyo Center, Israel; INFN, Italy; MEXT and JSPS, Japan; CNRST, Morocco; NWO, Netherlands; RCN, Norway; MNiSW and NCN, Poland; FCT, Portugal; MNE/IFA, Romania; MES of Russia and NRC KI, Russian Federation; JINR; MESTD, Serbia; MSSR, Slovakia; ARRS and MIZ\v{S}, Slovenia; DST/NRF, South Africa; MINECO, Spain; SRC and Wallenberg Foundation, Sweden; SERI, SNSF and Cantons of Bern and Geneva, Switzerland; MOST, Taiwan; TAEK, Turkey; STFC, United Kingdom; DOE and NSF, United States of America. In addition, individual groups and members have received support from BCKDF, the Canada Council, CANARIE, CRC, Compute Canada, FQRNT, and the Ontario Innovation Trust, Canada; EPLANET, ERC, ERDF, FP7, Horizon 2020 and Marie Sk{\l}odowska-Curie Actions, European Union; Investissements d'Avenir Labex and Idex, ANR, R{\'e}gion Auvergne and Fondation Partager le Savoir, France; DFG and AvH Foundation, Germany; Herakleitos, Thales and Aristeia programmes co-financed by EU-ESF and the Greek NSRF; BSF, GIF and Minerva, Israel; BRF, Norway; CERCA Programme Generalitat de Catalunya, Generalitat Valenciana, Spain; the Royal Society and Leverhulme Trust, United Kingdom.

The crucial computing support from all WLCG partners is acknowledged gratefully, in particular from CERN, the ATLAS Tier-1 facilities at TRIUMF (Canada), NDGF (Denmark, Norway, Sweden), CC-IN2P3 (France), KIT/GridKA (Germany), INFN-CNAF (Italy), NL-T1 (Netherlands), PIC (Spain), ASGC (Taiwan), RAL (UK) and BNL (USA), the Tier-2 facilities worldwide and large non-WLCG resource providers. Major contributors of computing resources are listed in Ref.~\cite{ATL-GEN-PUB-2016-002}.
%----------------------------------------------



The \texttt{atlaslatex} package contains the acknowledgements that were valid 
at the time of the release you are using.
These can be found in the \texttt{acknowledgements} subdirectory.
When your ATLAS paper or PUB/CONF note is ready to be published,
download the latest set of acknowledgements from:\\
\url{https://twiki.cern.ch/twiki/bin/view/AtlasProtected/PubComAcknowledgements}


%-------------------------------------------------------------------------------
\clearpage
\appendix
\part*{Appendix}
\addcontentsline{toc}{part}{Appendix}
%-------------------------------------------------------------------------------

In a paper, an appendix is used for technical details that would otherwise disturb the flow of the paper.
Such an appendix should be printed before the Bibliography.

%-------------------------------------------------------------------------------
% If you use biblatex and either biber or bibtex to process the bibliography
% just say \printbibliography here
\printbibliography
% If you want to use the traditional BibTeX you need to use the syntax below.
%\bibliographystyle{bib/bst/atlasBibStyleWoTitle}
%\bibliography{LFCalib_conf,bib/ATLAS,bib/CMS,bib/ConfNotes,bib/PubNotes}
%-------------------------------------------------------------------------------


%-------------------------------------------------------------------------------
\clearpage
\part*{Auxiliary material}
\addcontentsline{toc}{part}{Auxiliary material}
%-------------------------------------------------------------------------------

In an ATLAS paper, auxiliary plots and tables that are supposed to be made public 
should be collected in an appendix that has the title \enquote{Auxiliary material}.
This appendix should be printed after the Bibliography.
At the end of the paper approval procedure,
this information should be split into a separate document -- see \texttt{atlas-auxmat.tex}.

\end{document}
